\documentclass[a4paper]{article}

\usepackage[english]{babel}
\usepackage[utf8]{inputenc}
\usepackage{amsmath}
\usepackage{amssymb}
\usepackage{amsthm}
\usepackage{MnSymbol}
\usepackage{tikz-cd}
\usepackage{graphicx}
\usepackage{mathrsfs}
\usepackage[colorinlistoftodos]{todonotes}

\title{Discriminants and Quasi-symmetry}

\author{Alex Kite}

\date{\today}

\theoremstyle{plain}
\newtheorem{thm}{Theorem}[section]
\newtheorem{lem}[thm]{Lemma}
\newtheorem{cor}{Corollary}[thm]
\newtheorem*{thm*}{Theorem}
\newtheorem*{cor*}{Corollary}

\theoremstyle{definition}
\newtheorem{df}[thm]{Definition}
\newtheorem{eg}[thm]{Example}
\newtheorem{df/thm}[thm]{Definition/Theorem}

\theoremstyle{remark}
\newtheorem{rem}[thm]{Remark}
\newtheorem*{ack}{Acknowledgements}

\begin{document}
\maketitle

\section{Introduction}

Recently the notion of a quasi-symmetric representation (see \ref{QS}) has appeared both in \v{S}penko-Van-den-Bergh's construction of non-commutative crepant resolutions (NCCRs) of quotient singularities \cite{SV} and in Halpern-Leistner-Sam's construction of derived equivalences via toric variation of GIT (VGIT) and categorical actions of fundamental groupoids \cite{HLS}. Constructing NCCRs and fundamental groupoid actions via VGIT is hard/impossible in general. Yet in both these papers, quasi-symmetry means that such constructions are possible and combinatorially feasible. This brief note arose from an attempt to understand why this representation-theoretic condition simplifies the story so much. We explain this through the geometry of the ``discriminant locus" (see \ref{FIPS}) inside the secondary toric variety associated to the (maximal) torus representation. This discriminant locus may have several components but one of them is the so-called ``primary component" $\nabla_{pr}$ (see \ref{Pr}). Our main result characterises the quasi-symmetry condition in terms of $\nabla_{pr}$:

\begin{thm*}
A torus representation $T \lcirclearrowright V$ (whose determinant representation is trivial) is quasi-symmetric if and only if, either $T$ is rank 1 or else $\nabla_{pr}$ is not a divisor.
\end{thm*}

This quickly implies:

\begin{cor*}
The discriminant locus of a quasi-symmetric representation is a hyperplane arrangement (in log coordinates).
\end{cor*}
 
We hope that this will motivate interest in the non-quasi-symmetric case where the geometry of the discriminant is richer and the algebra of NCCRs and toric VGITs is harder.

In Section 2, we recall the theory of discriminants and determinants in the toric setting from \cite{GKZ}. Proofs of the main results are the content of Section 3.

\begin{ack}
I am indebted to Ed Segal for raising this question and to both him and Tom Coates for numerous useful discussions.
\end{ack}

\emph{Notation:}
We work in the usual toric setup (over $\mathbb{C}$) associated to a torus $T_L$ (where $L$ is the lattice of cocharacters of $T_L$) acting on $\mathbb{C}^n$ with weight (aka charge) matrix $Q$. We define $k:=\text{rank}(L)$, ignore any 0 weights and assume $Q$ is of full rank. Up to orbifolding by the finite group $L^{\vee}/\text{Im}(Q)$, we have a pair of short exact sequences, where $N$ is a lattice of rank $n-k$ and $M:=N^{\vee}$:
\begin{align}
0 & \rightarrow L  \xrightarrow{Q^{\vee}} \mathbb{Z}^n  \xrightarrow{A} N   \rightarrow 0 \nonumber \\
0 & \rightarrow M  \xrightarrow{A^{\vee}} (\mathbb{Z}^n)^{\vee}  \xrightarrow{Q} L^{\vee}  \rightarrow 0 
\label{ToricData}
\end{align}
If we prefer, we can think in terms of the cokernel $A$ (the \textit{ray map}) which gives the starting data (i.e. the rays) for a toric VGIT (see e.g. \cite{DH} for more details). We assume that these rays are distinct. In this setting, the weights can be assembled into a fan (the ``secondary fan") on $L^{\vee}_{\mathbb{R}}$ whose chambers (i.e. maximal cones) parametrise all the different projective simplicial fans supported on a subset of the original rays in $N$ (see e.g \cite{DH} or \cite{CLS} Ch. 14). Throughout, we denote rays by $\omega$ and weights by $\beta$.

%The data of $Q$ gives the rays of a canonical fan (called the secondary fan) defining the secondary toric variety $Y_A$, which is a partial compactification of $T_{L^{\vee}}$.

We work in the Calabi-Yau case  where $T_L$ acts through $SL_n(\mathbb{C})$ or, equivalently, the cone $\sigma$ generated by the rays ($=\text{A}_{\mathbb{R}}(\mathbb{R}_{\geq 0}^n)$) is Gorenstein. In this setting, we may pick $m \in M$ such that all rays lie in the affine hyperplane $H$ given by $\langle m,- \rangle =1$. This allows us to describe fan structures on $A$ in terms of the polytope $\Delta:=\sigma \cap H$. 
%In addition, the secondary fan is, in fact, compact in this case. 


\section{Discriminants in the toric setting}

We now introduce the principal A-determinant and Horn uniformization following \cite{GKZ}, Ch. 9 and 10. Let $A \subset N$ with $|A|=n$ be the image under $A$ of the standard basis in $\mathbb{Z}^n$ (i.e. the rays of our VGIT). Picking a basis for $N$ allows us to identify elements $n \in N$ with characters $x^{\omega}:=\prod x_i^{\omega_i}$ and hence to consider $\mathbb{C}^{A}:=\{f(x)=\sum_{\omega \in A} a_{\omega} x^{\omega} \}$, the set of all Laurent polynomials with exponents in A. 
\begin{df}{(\cite{GKZ}, Ch. 9, 1.2)}
$\nabla_0:=\{ f(x)=\{a_{\omega}\}_{\omega \in A} \in \mathbb{C}^A | \exists x_0 \in (\mathbb{C}^*)^{n-k}$ such that $\frac{\partial f}{\partial x_i}(x_0)=0$ for all $i \}$. $\nabla_A:=\bar{\nabla}_0$. When $\nabla_A$ is a hypersurface, its defining equation is defined to be the \textit{A-discriminant} $\Delta_A(\{a_{\omega}\}_{\omega \in A})$
\end{df}

\begin{df}{(\cite{GKZ}, Ch. 10, 1A)}
$\nabla'_0:=\{ (f_i(x))_i \in \prod_{i=1}^{n-k} \mathbb{C}^A | \exists x_0 \in (\mathbb{C}^*)^{n-k}$ such that $f_i(x_0)=0$ for all $i \}$. $\nabla'_A:=\bar{\nabla}'_0$. When $\nabla'_A$ is a hypersurface, its defining equation is the \textit{A-resultant} $R_A(\{f_i\}_{i=1, \hdots, n-k})$. The \textit{principal A-determinant} $E_A(f):=R_A(\{x_i \frac{\partial f}{\partial x_i}\}_{i=1, \hdots, n-k})$.
\end{df}

\begin{rem}
A na\"ive dimension count might suggest that $\nabla'_A$ is not a hypersurface for generic $A$. However, as $f(x) \in \mathbb{C}^A$ is quasi-homogeneous, in fact it should be.
\end{rem}

\begin{rem} 
A priori, $\nabla_A$ and $\{E_A=0\}$ are defined inside $(\mathbb{Z}^n)^{\vee} \otimes \mathbb{C}$ but, in fact, they have $k$ quasi-homogeneities (\cite{GKZ}, Ch. 9, 3.B) meaning that they descend to $T_{L^{\vee}}$. Whenever we write $\nabla_A$ or $\{E_A=0\}$ (unless specified otherwise), we shall mean this ``reduced" A-discriminant/A-determinant. 
\end{rem}

\begin{df}
The \textit{discriminant locus} is the subset $\{E_A=0\} \subset T_{L^{\vee}}$. The \textit{Fayet-Iliopoulos parameter space} (or \textit{FIPS} for short) of the VGIT defined by $A$ is the complement of the discriminant locus inside $T_{L^{\vee}}$.
\label{FIPS} 
\end{df}

\begin{rem}
We often work on the cover of the FIPS associated with the inclusion FIPS $\subset T_{L^{\vee}}$. If we pick coordinates on $T_{L^{\vee}}$, this amounts to working in the corresponding logarithmic coordinates, so we prefix the pullback of any object defined on $T_{L^{\vee}}$ under this cover by \textit{log}.
\end{rem}

\begin{rem}
It is helpful to `think of the discriminant locus as a complexification (or detropicalisation) of the VGIT wall and chamber structure (i.e. secondary fan) on $L^{\vee}_{\mathbb{R}}$ in the sense that the secondary fan is the normal fan of the Newton polytope of $E_A$ (\cite{GKZ}, Ch. 10, Thm 1.4).
\end{rem}

In general, despite $R_A(\{f_i\})$ being irreducible, this discriminant locus is reducible, with components (some of which may still be trivial) naturally indexed by non-empty faces $\Gamma \subset \Delta$. More precisely, 
\begin{thm} [\cite{GKZ}, Ch. 10, Thm 1.2] $E_A$ has prime factorisation $\prod_{\Gamma \subset \Delta} \Delta_{A \cap \Gamma}^{m(\Gamma)}$ where $m(\Gamma)$ is some multiplicity $\in \mathbb{N}$.
\label{Factor}
\end{thm}
Here $E_A$ is a function of $\{a_{\omega}\}_{\omega \in A}$ and we interpret $\Delta_{A \cap \Gamma}$ as such by pullback under the natural projection $\mathbb{C}^A \twoheadrightarrow \mathbb{C}^{A \cap \Gamma}$. 

\begin{df}
The \textit{primary component} of the discriminant locus is $\nabla_{pr}:=\nabla_A \subset T_{L^{\vee}}$. 
\label{Pr}
\end{df}

Theorem \ref{Factor} reveals the inductive nature of the A-determinant and forces us to consider the sub-VGIT problem associated to $\Gamma \subset \Delta$. One can check that 3 copies of the short exact sequence (\ref{ToricData}) fit together in a commutative diagram with exact rows and columns, where $n_{\Gamma}$ is the number of rays in the face $\Gamma$:



\begin{equation}
\begin{tikzcd}
\hspace{1cm} & 0 \arrow{d}   & 0 \arrow{d} & 0 \arrow{d} &  \\
0 \arrow{r} & M_{\Gamma}'  \arrow{d} \arrow{r} & (\mathbb{Z}^{n-n_{\Gamma}})^{\vee} \arrow{d} \arrow{r}  & (L'_{\Gamma})^{\vee} \arrow{d} \arrow{r}  & 0  \\
0 \arrow{r} & M \arrow{d} \arrow{r} & (\mathbb{Z}^n)^{\vee} \arrow{d} \arrow{r} & L^{\vee} \arrow{d}{p} \arrow{r} & 0 \\
0 \arrow{r} & M_{\Gamma} \arrow{d} \arrow{r} & (\mathbb{Z}^{n_{\Gamma}})^{\vee} \arrow{d} \arrow{r} & L_{\Gamma}^{\vee} \arrow{d} \arrow{r} & 0 \\	
 & 0 & 0 & 0 &   
\end{tikzcd}
\label{CommSquare}
\end{equation}

\begin{df/thm}{(\cite{GKZ},Ch. 9, 3.C)}
The \textit{Horn uniformization} is the rational map with image $\nabla_{pr}$ given by:
\begin{align*}
\mathbb{P}(L \otimes \mathbb{C}) & \dashedrightarrow \nabla_{pr} \subset T_{L^{\vee}}=\text{Hom}(L,\mathbb{C}^*) \\
{[a_1,\hdots ,a_n]} & \mapsto ((b_1, \hdots , b_n) \mapsto \prod_{i=1}^n a_i^{b_i} )
\end{align*} 
where here we identify $L$ with its image inside $\mathbb{Z}^n$. In the case when $\nabla_{pr}$ is a hypersurface, this is a birational parameterisation.
\end{df/thm}

If we pick a basis for $L$ given by $q_1, \hdots, q_k$ (with the components of $q_i$ in $\mathbb{Z}^n$ given by $q_{ij}$) and corresponding coordinates $\lambda_1, \hdots, \lambda_k$, then (identifying $T_{L^{\vee}} \cong (\mathbb{C}^*)^k$) we may rewrite this as:

\begin{align}
\mathbb{P}^{k-1} & \dashedrightarrow \nabla_{pr} \subset (\mathbb{C}^*)^k \nonumber \\ 
{[\lambda_1, \hdots, \lambda_k]} & \mapsto ( \prod_{j=1}^n ( \lambda_1 q_{1j}+ \hdots + \lambda_k q_{kj} )^{q_{ij}}  )_{i=1, \hdots, k} \label{Horn}
\end{align}

\section{Quasi-symmetry and discriminants}

\begin{df}{(\cite{SV}, 1.6)}
A torus representation $T_L \lcirclearrowright V$ with weights $\beta_i \in L^{\vee}$ is \textit{quasi-symmetric} if, for every line $l \subset L^{\vee}_{\mathbb{R}}$, $\sum_{i | \beta_i \in l} \beta_i=0$.
\label{QS}
\end{df}

\begin{rem}
A quasi-symmetric representation is necessarily Calabi-Yau, since the latter condition means $\sum_{i=1}^n \beta_i=0$. In fact, quasi-symmetry says that the $\mathbb{C}^*$-VGIT  of weights coming from any 1-parameter-subgroup of $T_L$ is Calabi-Yau.
\end{rem}

Following \cite{GKZ}, we introduce the terminology:
\begin{df}
A collection of rays $\{\omega_i\} \subset N_{\mathbb{R}}$ form a \textit{circuit} if there is precisely one linear relation between them. A face $\Gamma \subset \Delta$ is called a \textit{circuit} if the collection of all rays lying in $\Gamma$ forms a circuit.
\end{df}

\begin{thm}
The representation $T_L \lcirclearrowright \mathbb{C}^n$ is quasi-symmetric if and only if the Horn uniformization of $\nabla_{pr}$ is constant i.e.  $\nabla_{pr}$ is not a divisor except when $T_{L^{\vee}}$ is rank one, in which case $\nabla_{pr}$ is a point. 
\label{Main}
\end{thm}

\begin{proof}
From \eqref{Horn2}, the Horn uniformization is constant precisely when, for all $i$,
\begin{equation*}
\prod_{j=1}^n ( \lambda_1 q_{1j}+ \hdots + \lambda_k q_{kj} )^{q_{ij}}=\prod_{j=1}^n ( \lambda_1 \beta_{j1}+ \hdots + \lambda_k \beta_{jk} )^{\beta_{ji}} 
\end{equation*}
is constant as a degree 0 element of $\mathbb{C}(\lambda_1, \hdots, \lambda_k)$, where $\beta_{ji}$ are the $k$ components of $\beta_j \in L^{\vee}$.
Since $\sum_{m=1}^k \lambda_m \beta_{jm}$ cancels with $\sum_{m=1}^k \lambda_m \beta_{Jm}$ if and only if $\beta_j$ and $\beta_{J}$ lie on the same line in $L^{\vee}_{\mathbb{R}}$, decomposing $\prod_{j=1}^n ( \lambda_1 \beta_{j1}+ \hdots + \lambda_k \beta_{jk} )^{\beta_{ji}}$ as $\prod_{l \subset L^{\vee}_{\mathbb{R}}} (\prod_{j| \beta_j \in l} (\sum_{m=1}^k \lambda_m \beta_{jm})^{\beta_{ji}}$ shows that this is constant if and only if each factor $\prod_{j| \beta_j \in l} (\sum_{m=1}^k \lambda_m \beta_{jm})^{\beta_{ji}}$ is constant for all $i$ and lines $l$. Fix a primitive generator $\underline{l}=(l_1, \hdots, l_k)$ for $l$ and write each $\beta_j$ on $l$ as $n_j \underline{l}$. Then
\begin{gather*}
\prod_{j| \beta_j \in l} (\sum_{m=1}^k \lambda_m \beta_{jm})^{\beta_{ji}}=(\prod_{j| \beta_j \in l} 
n_j^{\beta_{ji}})(\sum_{m=1}^k \lambda_m l_{m})^{\sum_{j| \beta_j \in l} \beta_{ji}}  
\end{gather*}
is constant if and only if $\sum_{j| \beta_j \in l} \beta_{ji}=0$. Hence the claim for all $i$ and $l$ is precisely the quasi-symmetry condition.
\end{proof}



\begin{cor}
The log-discriminant locus associated to a quasi-symmetric representation is an (affine) hyperplane arrangement, whose hyperplanes are the log-$A \cap \Gamma$-discriminants arising from the faces $\Gamma \subset \Delta$ which are circuits.
\label{Cor}
\end{cor}

\begin{lem}
If the representation $T_L \lcirclearrowright \mathbb{C}^n$ is quasi-symmetric and $\Gamma \subset \Delta$ is a face, then the induced representation $T_{L_{\Gamma}} \lcirclearrowright \mathbb{C}^{n_{\Gamma}}$ is quasi-symmetric also.
\label{Inherit}
\end{lem}

\begin{proof}{(Lemma \ref{Inherit})}
Fix a line $\hat{l} \subset (L_{\Gamma})_{\mathbb{R}}^{\vee}$ and consider $\sum_{i| \hat{\beta}_i \in \hat{l}} \hat{\beta}_i$ where $\hat{\beta}_i:=p(\beta_i) \in L_{\Gamma}^{\vee}$ for $i$ such that $\omega_i \in \Gamma$. Take a term $\hat{\beta}_i$ in this sum and consider the natural lift $\beta_i \in L^{\vee}$. It defines a line $l \subset L^{\vee}_{\mathbb{R}}$ (recall we are ignoring 0 weights) and so, by quasi-symmetry, $\sum_{i| \beta_i \in l} \beta_i=0$. Since $l \nsubset (L'_{\Gamma})_{\mathbb{R}}^{\vee}$, the commutative diagram \eqref{CommSquare} implies that all of the $\beta_i$ in this sum correspond to rays in $\Gamma$ and hence project under $p$ to give some of the remaining $\hat{\beta}_i$ in our sum. So we've proved the sub-term $\sum_{i | \beta_i \in l} \hat{\beta}_i=0$. Iterating this procedure yields the desired conclusion.
\end{proof}

\begin{proof}{(Corollary \ref{Cor})}
The discriminant $\Delta_{A \cap \Gamma}$ in $E_A$ is a function only of $a_{\omega}$ where $\omega \in \Gamma \cap A$. Moreover, if $\Gamma$ is a circuit, then \cite{GKZ}, Ch. 9, Prop 1.8 says that $\Delta_{A \cap \Gamma}$ is an affine (log-)hyperplane. 

So we need to show that if $\Gamma$ is not a circuit, then it doesn't contribute to $E_A$. In this case, the space $L_{\Gamma}$ of relations in $\Gamma$ has $\text{dim}(L_{\Gamma})>1$, hence we are done by Lemma \ref{Inherit} and Theorem \ref{Main}.     
\end{proof}
 
 
\begin{thebibliography}{9}

\bibitem{CLS} D. Cox, J. Little, H. Schenck, \textit{Toric Varieties}, Graduate Studies in Mathematics, Vol. 124, 2011

\bibitem{DH} I. Dolgachev, Y. Hu, \textit{Variation of Geometric Invariant Theory Quotients}, arXiv:9402008

\bibitem{GKZ} I. Gelfand, M. Kapranov, A. Zelevinsky, \textit{Discriminants, Resultants, and Multidimensional Determinants}, Birkh\"auser Boston, 1994

\bibitem{HLS} D. Halpern-Leistner, S. Sam, \textit{Combinatorial Constructions of Derived Equivalences}, arXiv:1601.02030

\bibitem{SV} \v{S}. \v{S}penko, M. Van den Bergh, \textit{Non-commutative resolutions of quotient singularities},	arXiv:1502.05240

\end{thebibliography}
\end{document}